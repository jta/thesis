\begin{acknowledgements}
\addcontentsline{toc}{chapter}{Acknowledgements}
%The resulting interactions between congestion control and traffic engineering are not easily understood, and have led to a more opaque and convoluted Internet.


%In this paper we debate existing approaches to resource pooling and present PREFLEX, an architecture where edge networks and hosts both share the burden and reap the rewards of balancing traffic over multiple paths. 
%Using PREF (Path RE-Feedback), networks suggest outbound paths to hosts, who in turn use LEX (Loss Exposure) to signal transport layer semantics such as loss and flow start to the underlying network. 
%By making apparent network preferences and transport expectations, PREFLEX provides a mutualistic framework where congestion control and traffic engineering can both coexist and evolve independently.
%

%In particular such a system must rarely reorder packets, must not require per-flow state, must cope with different paths having different bandwidths and must be self-tuning in a variety of network contexts. 
%
%
%This paper argues there is a need to rethink the current traffic management architecture to involve both network and transport layer, allowing end-host congestion control and network traffic engineering to become aware of each other. 
%Only by fomenting cross layer cooperation will traffic management be able to provide the robustness and performance required for the future Internet without compromising its scalability and capacity for innovation.
%

\end{acknowledgements}

