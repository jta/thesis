% Abstract
\begin{abstract}
\addcontentsline{toc}{chapter}{Abstract}

Parallelism pervades the Internet, yet efficiently pooling this increasing path diversity has remained elusive. 
With no holistic solution for resource pooling, each layer of the Internet architecture attempts to balance traffic according to its own needs, potentially at the expense of others.
From the edges, traffic is implicitly pooled over multiple paths by retrieving content from different sources.
Within the network, traffic is explicitly balanced across multiple links through the use of traffic engineering.
This work explores how the current architecture can be realigned to facilitate resource pooling at both network and transport layers, where tension between stakeholders is strongest.

The central theme of this thesis is that \emph{traffic engineering} can be performed more efficiently, flexibly and robustly through the use of \emph{re-feedback}.
A cross-layer architecture is proposed for sharing the responsibility for resource pooling across both hosts and network.
Building on this framework, two novel forms of traffic management are evaluated.
Efficient pooling of traffic across paths is achieved through the development of an in-network congestion balancer, which works even in the absence of multipath transport.
Network and transport mechanisms are then designed and implemented to facilitate path fail-over, greatly improving resilience even in the absence of receiver side cooperation.
These contributions are framed by a longitudinal measurement study which provides evidence for many of the design choices taken.
A methodology for scalably recovering flow metrics from passive traces is developed and systematically applied to over five years of interdomain traffic data.
The resulting findings challenge traditional assumptions on the preponderance of congestion control on resource sharing, with over half of all traffic being constrained by limits other than network capacity.

All of the above represent concerted attempts to rethink and reassert traffic engineering in an Internet where competing solutions for resource pooling proliferate.
By delegating responsibilities currently overloading the routing architecture towards hosts and re-engineering traffic management around the core strengths of the network, the proposed architectural changes allow the tussle surrounding resource pooling to be drawn out without compromising the scalability and evolvability of the Internet.


\end{abstract}

