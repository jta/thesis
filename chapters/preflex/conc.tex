\section{Closing the loop}

This chapter broadly describes an architecture which shares the responsibility for resource pooling between endhosts and edge networks, but does not explicitly dictate an outcome. \ac{PREFLEX} has been designed to take into account the inevitable tussle which will occur between both, and envisages use cases where control over resource pooling could feasibly shift entirely in one direction or the other. 

At its most liberal, \ac{PREFLEX} enables resource pooling to be entirely performed by end-hosts. At its most conservative, \ac{PREFLEX} affords edge network providers more fine-grained control over traffic than before. Between either extreme, the resulting mutualistic architecture offers greater transparency, control and robustness by realigning the interface between network and transport in order to accommodate the needs of both.

In order to make \ac{PREFLEX} deployable, the scope of the architecture has been restricted to stub domains.
While this necessarily reduces the amount of path diversity which can be explored, there are compelling reasons to abdicate some flexibility in favour of a more practical solution:

\begin{itemize}
\item{
    \textbf{Most benefit derives from a limited set of paths}.
    Work on source routing \cite{Gummadi:2004p131,Yang:2006p405} suggests most improvement in resilience can be extracted from a small set of deflections.
    Stub domains such as \acp{ISP} and entreprise networks are naturally multi-homed, and can provide reasonable path diversity given an adequate selection of providers.
    Expecting the establishment of cross-domain paths has been a pitfall for previous research in \ac{QoS} in particular and is marred by difficulties in providing incentives for all parties involved.
}
\item{
    \textbf{Stub domains are most aligned with user interests}. 
    For enterprise, academic and content distribution networks, end-hosts are typically managed by the same entity which operates the network.
    For \acp{ISP}, there is a binding contract between end-users and provider.
    In either case, the stub domain not only benefits from not deteriorating the quality of service provided to hosts locally, but also has an interest in making sure end-to-end traffic is unaffected by outages or degraded performance.
    The inability to reach a website is often misdiagnosed by users as a failure of the stub domain, rather than the intervening network path or the remote host.
    As a result, the burden of remote failures is often placed on local customer support.
    \ac{PREFLEX} allows resources not only to be shared more efficiently, but also potentially reduces the impact of remote network events on stub domains.
}

\item{
    \textbf{The Internet topology is flattening}. 
    In studying over 100 \acp{ISP}, transit providers and content providers for over two years, Labovitz \emph{et al.} \cite{Labovitz:2010p175} established the changing nature of the Internet topology, migrating from a hierarchy of providers to an increasingly interconnected model.
    The rise of \acp{IXP} where customer networks peer directly with content providers has had a profound impact in shaping traffic patterns and interdomain traffic in particular.
    The length of \ac{AS} paths has decreased for most \emph{content}.
    As a consequence, it has become simpler for a stub domain to ensure path disjointness as a larger portion of the end-to-end path is under its control.
}
\end{itemize}

The resulting architecture provides many benefits. 
Balancing is both transport agnostic and allows flow state to be kept to a minimum, requiring policing at the edges only if congestion accountability is required. 
Additionally, path selection is receiver driven, aligning the stakeholder who can decide when to issue \ac{FNE} packets with the stakeholder who benefits the most. 
The key to path selection is the information provided by the sender (point 7 in figure \ref{fig:preflexout} ), which allows the network to estimate path loss on the same timescale as hosts. 
By pooling loss information from multiple hosts, the network may additionally be made aware of path failures sooner than hosts using \ac{TCP} inference.

%We assume for the entirety of this section that each flow follows a single path and that a host does not override the path selected by its network, although neither is strictly necessary. 
%The implications of both assumptions being broken and the incentives involved are briefly reviewed in \cite{Araujo:2010p224}.
