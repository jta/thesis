\section{Related work}
\label{section:malawi:related}

Despite their inherent value, longitudinal studies of Internet phenomena are rare. 
Over its short lifespan the Internet has been shaped as much by technological change as by political and commercial realities. 
This dynamic nature does not lend itself to observational studies where data must be collected and curated over long periods of time, and has resulted in a scarcity of relevant datasets. 
What few exceptions exist often stem from collaborative research efforts, such as CAIDA \cite{CAIDA} or Oregon Routeviews \cite{routeviews}. 
The usefulness of these datasets however can be severely affected by the need for data privacy. 
The dissemination of interdomain routing information, where no such requirement exists, has assisted in a wealth of research on wide ranging topics, from quantifying path diversity \cite{Oliveira:2009p203} to locating Internet bottlenecks \cite{Hu:2004p96}. 
In contrast, longitudinal datasets relating to passive measurements have nurtured a much smaller community of researchers often focusing on characterizing traffic \cite{Fontugne:2010p413}. 
Stripped of the locality contained within IP addresses however, researchers are left unable to relate these findings to a wider context.
Instead, cross-sectional studies characterizing traffic aggregated by location are frequently conducted under different contexts \cite{Ager:2011p528}, but lack the temporal perspective only longitudinal studies can afford. 
Efforts to characterize the spatial properties of traffic over time \cite{Dhamdhere:2011p428,Labovitz:2010:IIT:2043164.1851194,Cho:2008p488} have defined the changing of Internet topology and traffic alike but fall short of relating such shifts with their impact on relevant metrics such as loss or delay. 

% other work since
This chapter builds on a wealth of prior work on understanding Internet traffic and serves as a reappraisal of significant past contributions.
Flow characteristics and \ac{TCP} behaviour at large are subject to frequent reassessment \cite{Zhang:2002p85}.
Of particular relevance to the current work are passive studies which delve into the inner mechanisms of \ac{TCP}.
In \cite{Jaiswal:2004p242}, Jaiswal et al.\ infer the sender's congestion window by identifying the congestion control variant from the behaviour observed during loss recovery.
The use of separate state machines for each variant however proves unscalable given the many flavours of \ac{TCP} congestion control which have since been deployed.
In \cite{Lan:2006p566}, Lan et al.\ analyse flows according to size, duration, rate and burstiness and characterise the observed correlations for heavy-hitters specifically,
uncovering evidence of increased application influence on flow rates and burstiness and consequently suggest treating flow size and duration as independent dimensions.

One central aspect to the analysis of \ac{TCP} behaviour is the estimation of \ac{RTT} from packet capture data. 
In addition to SYN-based methods, Shakkotai et al.\ \cite{Shakkottai:2004p408} evaluate further techniques to estimate the \ac{RTT} of a unidirectional flow. 
The \textit{rate change} method establishes a relation between the \ac{RTT} and the increase in sending rate, assuming linear window increases during congestion avoidance. 
Unfortunately, this assumption no longer holds, both due to the proliferation of less conservative congestion control algorithms such as CUBIC \cite{Ha:2008p471}, and due to application-driven flow control. 
An alternative is the use of frequency-domain techniques \cite{Veal:2005p412,Lance:2005p565,Qian:2009p429}, which are a natural fit given the self-clocking nature of \ac{TCP}. 
However, a common difficulty with the application of spectral analysis is extracting the fundamental frequency which corresponds to the \ac{RTT} in the presence of noise. 
In applying the Fourier transform to inter-packet arrival times, for example, Qian et al.\ \cite{Qian:2009p429} note that less than half of all flows have distinguishable \textit{flow clocks}; likewise, the \ac{FFT}-based \ac{RTT} recovery was found to be unreliable even after pre-processing available data to enhance inherent periodicities.

% topological influence
Finally, it is important to elucidate what changes in traffic properties are intrinsic to \ac{TCP} and data transfer, and which ones arise from large-scale changes in the \ac{AS}-level topology of the Internet. 
In the decade since publication of \cite{Zhang:2002p85}, the Internet has undergone significant changes, shifting from a broadly hierarchical form to a flatter, more interconnected structure \cite{Labovitz:2010p175,Ager:2012p567}.
Given the longitudinal nature of this chapter and its focus on interdomain traffic in particular, the insights provided by these studies on the macroscopic effects of content consolidation are discernible within the studied dataset, and as such are a source of validation for many of the observations herein.
