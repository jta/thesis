By readjusting traffic according to end-to-end metrics, \ac{PREFLEX} is unique in proposing congestion, rather than just load, as an essential metric for traffic engineering.
In the previous chapter necessarily artificial end-to-end behaviour was used in order to gain insight into how \ac{PREFLEX} works.
Understanding the extent to which \ac{PREFLEX} can benefit end-users and networks in practice however requires a deeper understanding of the inherent characteristics of Internet traffic at large.

This chapter provides a longitudinal analysis of the characteristics of end-to-end Internet traffic, describing the shifting trends in interdomain traffic as viewed from \acs{WIDE}, a Japanese academic provider.
Over time, this vantage point is subject to upgrades, changes in routing policy and congestion events, all of which can hinder the interpretation of data.
These limitations are overcome by looking further afield, searching for clues within shifts in the geographical and topological make-up of inbound and outbound traffic and how these trends relate to end-to-end performance.
