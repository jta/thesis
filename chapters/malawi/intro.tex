Providing a macroscopic view on where traffic originates from, and in what quantity, can be achieved by simply binning packets into flows and accumulating byte counts over geographical or topological locations. 
Uncovering application layer characteristics (i.e. how traffic is sent) is a more complex problem that requires additional methods to reverse engineer transport behaviour.
This chapter builds on the analysis of the \ac{MAWI} dataset presented in chapter \ref{chapter:malawi}, focusing on the longitudinal evolution of \ac{TCP} behaviour.
\ac{TCP} plays a central role in mediating between application needs and network capacity.
In the absence of congestion, \ac{TCP} is responsible for increasing sending rates in a bid to make efficient use of available bandwidth.
Conversely, should congestion arise, \ac{TCP} is expected to reduce its rate.
This form of congestion control embedded in \ac{TCP} is largely credited with performing resource allocation across the Internet and averting congestion collapse.
But to what extent does this behaviour describe \ac{TCP} throughput in practice?

% app limited
For some types of traffic, throughput is not strictly dictated by the outcome of congestion control.
For one, inelastic traffic such as streaming media typically has bounds on the amount of capacity required.
Beyond a certain throughput rate, bandwidth probing by \ac{TCP} is often unnecessary and occasionally harmful.
Since \ac{TCP} drives itself relentlessly towards congestion, real-time applications may suffer from increased latency and jitter.
Furthermore, content providers may wish to avoid exceeding the streaming rate for content which may not be consumed in its entirety due to user behaviour, for example channel hopping for video streaming services \cite{iptvWorkload}, or client limitations, such as buffering restrictions on mobile devices \cite{Rao:2011p547}.
% one click?
Such forms of \emph{application pacing} are often also applied to elastic traffic.
In some cases, high throughput is perceived and subsequently marketed as a value added service.
One-click hosting services such Rapidshare and Megaupload \cite{oneclick1, SanjuasCuxart:2012p588} actively monetise access to large amounts of content both through online advertising and subscription models to bandwidth tiers.
Conversely, file sharing applications such as Bittorrent and other peer-to-peer clients allow users to rate limit transfers in order to reduce impact on competing traffic, or to provide incentives for participation \cite{bittorrentIMC}.

Even beyond such application behaviour, flows may still be constrained by factors not pertaining directly to \ac{TCP} congestion control.
The transport layer is subject to strict bounds within which it can operate, potentially impeded by socket buffer sizes set by \acf{OS} vendors or tuned by network administrators. 
There is an upper bound on the window size a receiver can advertise back to the sender; in the absence of windowscale negotiation \cite{braden1989rfc}, no \ac{TCP} connection can exceed a 64KB window. 
In addition, resource sharing is often subject to local policy.
In the absence of adequate methods for readjusting how \ac{TCP} distributes bandwidth, network operators and system administrators often trade efficiency for predictability, shaping traffic to conform to local notions of fairness or in anticipation for expected demand \cite{ispTrafficShaping}.

Given these different potential sources of rate control, \emph{what can be said about their relative impact on Internet traffic at large}?
Much of the underlying motivation is shared with the landmark study by Zhang et al.\ \cite{Zhang:2002p85} on the characteristics of Internet flow rates.
Using traces spanning both access, peering and regional links, Zhang et al.\ analyse traffic according to potential rate limiting factors.
Amongst other findings, host window limitations were found to affect over 30\% of traffic for the access networks studied.
Importantly, the authors found a strong correlation between flow throughput and flow size, postulating that this could derive from user behaviour, with large transfers more likely to be performed over higher bandwidth connections.

% application space.
In addition to such general investigations, this chapter is equally indebted to comprehensive work of a narrower scope.
Significant portions of the observed traffic pertain to well known applications which have been previously studied.
Rao et al.\ \cite{Rao:2011p547} survey strategies used for video streaming at both Youtube and Netflix and characterise the properties of interleaved \emph{block sending} patterns used to pace streams.
These patterns are also the subject of \cite{Alcock:2011p575}, in which the burstiness of Youtube traffic in particular is found to result in considerable losses over residential connections.
A large portion of the traffic observed in the \acs{MAWI} dataset originates from \ac{HTTP} file sharing services, commonly referred to as \acf{OCH} websites \cite{oneclick1}.
In \cite{SanjuasCuxart:2012p588}, the authors study the characteristics of such traffic over a three month period, detailing the different throttling strategies used by different providers.

