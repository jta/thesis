This chapter refines the concepts proposed \ac{PREFLEX} given the insights provided by the \ac{MAWI} dataset presented in chapters \ref{chapter:malawi} and \ref{chapter:rate}.
This evidence-based approach is further enhanced by adapting the resulting architecture, INFLEX, to function within existing frameworks for \ac{SDN}.
This chapter is organized as follows:

\renewcommand{\descriptionlabel}[1]{\hspace{\labelsep}\textbf{Section #1}}
\begin{description}
\item[\ref{section:inflex:design}] reviews results from the \ac{MAWI} dataset and their implications for the design of a resource pooling architecture.
\item[\ref{section:inflex:background}] provides an overview of \ac{SDN} and Openflow, a standardized protocol for interfacing between control and data plane.
\item[\ref{section:inflex:arch}] describes the design of INFLEX, a cross-layer architecture for network resilience.
\item[\ref{section:inflex:eval}] evaluates path fail-over and network overhead of the proposed solution
\item[\ref{section:inflex:discussion}] discusses how INFLEX can be extended from providing resilience alone into a unifying traffic management solution, encompassing most of the benefits put forward in chapters \ref{chapter:preflex} and \ref{chapter:cate}.
\end{description}


%Our contribution
%This chapter presents INFLEX, an \ac{SDN}-based architecture for cross-layer network resilience which provides on-demand path fail-over for \ac{IP} traffic. 
%The architecture is both unilaterally deployable, providing benefits even when adopted by individual domains, and inherently end-to-end, potentially covering third party failures.
%
%INFLEX operates by allowing an \ac{SDN}-enabled routing layer to expose multiple \emph{routing planes} to the transport layer. 
%Hence, traffic can be shifted by one routing plane to another as a response to end-to-end failure detection.
%Since INFLEX operates as an extension to the network abstraction provided by \ac{IP}, it can be used by all transport protocols.
%At the host, the proposed architecture allows transport protocols to switch network paths at a timescale which avoids flow disruption and which can be transparently integrated into existing congestion control mechanisms.
%Within the network, INFLEX provides both greater insight into end-to-end path quality, assisting fault detection, and more control over flow path assignment, enabling more effective fault recovery. 
%Modifications to both the \ac{TCP}/\ac{IP} network stack and a popular Openflow controller \cite{pox} are made publicly available.

